perational risk management is a core function on most large organisations
The risk assessment should identify the threats to assets, vulnerabilities and impacts on the organization and should determine the degree of risk
Every organisation is encouraged to choose the approach that is most applicable for its industry, complexity and risk environment
The risk assessment must be a formal process, planned, and all input data and their analyses should be recorded
Formal does not require risk assessment tools but such tools can make it less time consuming and more meaningful
The techniques employed to carry out the risk assessment should be consistent with the complexity and level of assurance required by the board
It is essential that the risk assessment should be done methodically systematically and comprehensively, producing comparable and reproducible results.
In cases where security is already in place these should also be controlled

A risk analysis should cover
\begin{itemize}
    \item How to eliminate risks
    \item Reduce risk
    \item Transfer them to a different organization(Insurance)
\end{itemize}
Risks can be an assessment of the economic benefits that can derive from an investment
The cost of of implementing something should be significantly outweighed by economic benefits or economic loss it prevents
The organisation uses risk acceptance criteria to determine what risks it is willing to tolerate.
Should be clearly decided (how much economic risk will be tolerated against how much it will cost to reduce said risk)
How likely is too likely against how fatal a risk could become.

Risk assessment papers are required in some places (UK) these include:
\begin{itemize}
    \item Health and safety display screens regulations
    \item Personal protective equipment work regulations
    \item Control of substances hazardous to health regulations
    \item Management of health and safety at work regulations
\end{itemize}
Who conduct the risk analysis:
\begin{itemize}
    \item Up to each individual organisation
    \item Annual risk assessments should be taken into consideration(someone with an updated view of threats and vulnerabilities)
    \item Changes in the company will need to be assessed
    \item Legislation, regulation and society changes
    \item Trusted qualified and experienced people
    \item Existing security/risk management roles in the organisation
    \item People with existing knowledge of current risks
    \item Can train someone internally for risk management, internal is recommended due to annual reviews
    \item Or Hire an external, possibly with a multiyear contract
\end{itemize}

Quantitative Risk analysis(not recommended) consists of:
\begin{itemize}
    \item Probability of an event 
    \item Potential loss from the event in cash
    \item multiply these together to get the ALE(annual loss expectancy) or EAC(Estimated annual cost)
\end{itemize}

Qualitative Risk analysis By far more widely used Does not require numerical probability data
Only estimated potential loss is use Focuses on related risks, to find where most vulnerabilities and threats are found
Three damage categories:
\begin{itemize}
    \item Damage to the organization's business(competitive position, finances, reputation)
    \item Contractual commitments
    \item Legal responsibilities
\end{itemize}

Steps
\begin{itemize}
\item Assets: Identify all information assets and which role or department that owns it
\item Threats: What threats are there for each asset
\item Vulnerabilities: What can allow or make a threat possible and dangerous
\item Impacts: what impact will the a threat have, if possible give a monetary value
\item Risk assessment: Look at the overview and consider the likelihood of something happening, and decide what is acceptable
\item Controls: Are the countermeasure to keep all risks within an acceptable terms. These controls can be:
\item Directive: administrative controlsrols such as creating policies
\item Preventive: protect vulnerabilities Theseo make something less likely or reduce the impact
\item Detective: discover attacks and trigger preventative or corrective  controls
\item Corrective: Reduce the effect of an attack
\item Recovery: Business continuity and     disaster recovery
\end{itemize}
The cost of implementing a control should be no greater than the cost of the impact 
It is not possible to remove all risks 
\begin{itemize}
\item Scope: identify the boundaries of what is to be protected
\item What'Whats within the organization and what is outside
\item Boundaries are physically or logically identifiable
\item Witch networks, data, and locations networksed to be protected
   \item  Parts of the organisation within the scope must bee capable of separation from their parties and from other organisations within a larger group
   \item  Hard to implement ISMS for organisations that are not self contained, with its own board and directors, and control over its own network
   \item Possible for larger organisations to pursue the certification independently 
   \item All physical premises should be listed and their networks and information assets
   \item Identify the assets and all systems necessary to process them
   \item Information assets: Informationformation systems or a body of information, files and conversations 
   \item conversations IT systems, software(client relationship management system, payment system, mail etc.)
   \item IT systems, Hardware(Servers, workstations, network)
   \item Telecommunication systems
   \item Every asset have an owner who is responsibilitiesle, find these by position rather than name
   \item Identify the relations between the systems, the assets and the organisational objectives and tasks
   \item The key objectivestives with contractual or legal aspect to them should be identified in the organization's plan.
   \item objectives should be SMART : specific, measurable, acceptablehievable, realistic, time bound
   \item Find the key objectives and focus on the most important ones
   \item This is best done by the whole implementation team in one session 
   \item Find what systems that are critical for the organisation tasks and objectives
   \item Rank systems in order of critical priority
   \item Give exact measures to risk categories
   \item Find potential threats to critical systems 
   \item Request input from a trained information security specialist
   \item Can be external to the system but not necessarily the organisation
   \item Both intentional attackers and careless workers
   \item Consider links between vulnerabilitiesties and threats
   \item What is a threat to one system is not necessarily and threat to another one
   \item Categorize likelihood of occurrence 
   \item Find potentialotential vulnerabilities
   \item Security weaknesses in the system
   \item Vulnerabilities can be exploited by threats and can lead to 
\end{itemize}
Annex A of the standards have a number of controls to be considered and is the first document an inspector will want to see
\begin{itemize}
   \item It should be review regularly
   \item Also used to third parties to show what degree of security that has been implemented.
\end{itemize}
the SoA (Statement of Applicability) should form the core of an ISMS manual (Example page 94)
\begin{itemize}
   \item Works as key evidence of steps taken between risk assessment and implementation of appropriate controls
   \item Should not in itself contain sensitive information
   \item Increasing number of software tools to help automate risk management and generate the SoA
\end{itemize}
    Risk treatment plan
\begin{itemize}
   \item Identifies the appropriate management action, responsibilities and priorities for managing information security risks
   \item Clearly defined risk assessment process
   \item Include what controls are in place
   \item What controls where considered necessary and the timeframe for implementation
   \item Should include required competence and training and awareness necessary for execution and continuous improvement
   \item Should link all phases together (Plan,Do,Check,Act)
\end{itemize}
    Measures of effectiveness, Three questions should be answered
\begin{itemize}
   \item What is the objective of each control
   \item how can you determine if the control is effective
   \item What are the parameters that will give a positive indication of control effectiveness
\end{itemize}
    Measures of effectiveness can be time consuming and should be discussed with each implementation
\begin{itemize}
   \item Over-reliance on negative reporting is likely to result in flawed measures
   \item Automated monitoring is preferable to manual arrangements
   \item The exact aspect being measured needs to be aligned with the main objective
   \item The integrity of the measures or statistics being produced is of paramount importance, as management decisions are likely to be based on this information
\end{itemize}

