First step of an ISMS: Define the infosec policy (should aim for a short statement of max 2 pages)
\begin{itemize}
    \item Take into account, the organisation location, assets, technology
    \item Risk assessment
    \item Approved by managers
    \item Must be reviewed and updated regularly (annually)
\end{itemize}
Must answer:
\begin{itemize}
    \item Who(does this affect)
    \item Everything must be agreed upon by the board and the steering group,
    \item Should be broad so the management and security can act without consulting the board for changes
    \item Might include all employees
    \item Possibly customers, suppliers, shareholders and other third parties
    \item Where (scope of the policy)
    \item Must be stated clearly
    \item Might differ from site to statedite in multi-site operation or virtual organisations
    \item Clearly define    what parts of operation thats not included and their security implications
    \item What
    \item Availability
    \item Confidentiality
    \item Integrity
    \item What's important and how it is to be prioritized
    \item Why
    \item Protection of information from and wide range of threats
    \item Ensure business continuity, minimize business damage
    \item Maximize return on investments and business opportunities to maintain the competitive edge
    \item Cash flow profitability legal compliance and commercial image
    \item Should include what nature of threats faced by the organisation
    \item Should look up industry standard and local specific information
\end{itemize}

In addition to the policy statement there should be a plan containing
\begin{itemize}
    \item Benefits of an ISMS
    \item Cost
    \item Expected implementation time
    \item When progress reviews should be done(after risk analysis, after drafts, after base implementation, after system audits, and annually)
\end{itemize}
