Clause 5.2.1 of the standard requires the organization to provide appro-
priate and adequate resources to carry out all the Plan—Do—Check—Act
(PDCA) phases of information security management. Clause 5.2.2 requires
that whoever is assigned an ISMS-related task has the necessary competence.
These two clauses can be satisfied at the same time as the required controls are
constructed. It Will be necessary to demonstrate, in the documentation, how
the competences were determined, and why.

\subsection{Job descriptions and compentency requirements}
Should contain 1) a description of the competencies need 2) a statement that 
ever employee is requierd to be aware of the organizations info sec policy. 
Attetntions should be drawn to the responsibility to protect assets from unauth
acces, disclosure, modification, distruction or interference. Job description 
should set out The clearly that breach of information security controls may be 
considered a mis demeanour under the organization’s disciplinary policy and that 
breach of them might, under specific circumstances, result in dismissal.

\subsection{Screening}
Control A.8.1.2 of the standard requires the organization to carry out 
verification checks on permanent staff, contractors and third parties at the time of
job applications. The organization should identify who will be responsible for
carrying this out, how it will be done, how the data will be managed and who
will have what authority in respect of the data and the recruitment process.
For some roles criminal screening must be done. There are four (actually 5?) 
basic checks that should be completed.

\begin{enumerate}
    \item Character reference checks, one personal and one business. Prefrably 
        written, but might be a signed transcript of a phone call carried out
        by a person experienced with phone call reference checks.

    \item A completness and accuracy check of the CV, usually carried out by
        written references supplied by previous employers. It is critical that
        the employer is methodical in ensuring that all facts are true.

    \item  Confirmation of claimed academic and professional qualifications,
        either by means of obtaining from the candidate copies of the
        certificates or other statement of qualification or through an
        independent CV checking service.

    \item There should be an independent identity check against a passport or 
        similar document that shows a photograph of the employee.

    \item  Finally, the individual’s entitlement to live and work in the country
        should be confirmed, by reference to appropriately endorsed travel or
        work documents.
\end{enumerate}
A draft contract can be agreeded upon but not signed before the checks are 
completed. In some cases if the job only deals with low level of information
people can start work before checks are completed.\\

Organizations should have records for existing stadd of equivalent completness
to those requierd for new hiers. This process should be done open and quickly,
and staff should be aware of the process. If it is found that existing staff has
incorrect or false CVs the organization will have to judge the exten it 
threatens info sec. There needs to be a procedure in places that allows new 
and/or inexperienced staff to have access to sensitive systems under supervision.
The performance of staff that has access to sensitive information should be 
reviewed at least annually.

\subsection{Terms and conditions of employment}
employees, contractors and third parties all agree and sign an employment
contract that contains and conditions covering their and the orgs 
resposibilities for info sec. It should include a confidesiality agreement
that covers information acquired prior to and during employment.
tandard confidentiality agreement. If loopholes are found the documents should
be ammended, and if it is significat replace and re-sign existing confidentiality
agreements and NDAs. The contract should make it clear that the employee has a
resposibility for info sec. This resposibility must be described.

\subsection{During employment}
A organization has to ensure that employees, contractors, and thidr-party users
are aware of information security threats as well as their resposibilities and
liabilities, and that it has trained personell appropriately. ISO25002's 
includes ensuring that staff are: properly brifed on their roles and 
resposibilities before they are granted access to senstive information, or
information systems.(information security threats, risks, and vurneabilities) 
All staff must appropriate awareness training and other training, as well as 
regular updates and communications.\\

Any staff involved in handling payment card data, and working within a card—
holder data environment as defined by the PCI DSS, will also need specific
training on their responsibilities in regard to that data.\\
There are also a number of staff who will require other user—specific train—
ing. These include the staff identified at the beginning of this chapter as needing
specific statements in their job descriptions and contracts of employment
about their information security responsibilities. These include:
\begin{itemize}
    \item the chief information officer;
    \item the information security adviser;
    \item members of the information security management forum;
    \item IT managers;
    \item network managers;
    \item IT and helpdesk support staff;
    \item webmasters;
    \item premises security staff
    \item HR, recruitment and training staff;
    \item general managers;
    \item finance staff;
    \item the company secretary and legal staff;
    \item internal quality assurance or system auditors;
    \item business continuity and emergecy response teams.
    \item basically everyone except for the cleaning lady..
\end{itemize}
Clause 5.2.2 also requires the organization to maintain records of education, 
training, skills, experience and qualifications, and this requirement is
satisfied by following the recommendations of this chapter and attaching these
records to the individual’s personnel file. More importantly, the effectiveness of
the training must be evaluated, and this requires the specific objectives for
each piece of training, and the criteria for measuring its effectiveness, to be
identified and agreed in advance. This is in line with best practice for effective
staff training.

\subsection{Disciplinary process}
Employees that violate information security policies should be lashed 
accordingly. We will worry about finding/creating evidence of a breach later.

\subsection{Termination or change of employment}

