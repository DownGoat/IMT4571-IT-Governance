\section{Secure Areas}
Control A.9.1 of the standard deals with secure areas. Its objective is to prevent
unauthorized physical access, damage or interference to business premises
and information. It has six sub-clauses. Critical or sensitive information and
information processing facilities should be housed in secure areas protected
by a defined secure perimeter, with appropriate security barriers (eg walls, fixed
floors and ceilings, card-controlled entry gates) and controls (eg staffed recep-
tion desks) that provide protection against unauthorized access or damage to
papers, media or information processing facilities. The protection implemented
should be commensurate with the assessed risks and the classification of the
information, and should take into account out-of—hours working and similar
issues.

\subsection{Physical security perimeter}
Organizations are requierd to use a security perimeter to protect areas that 
contain information processing facilities. If the risk is there, more than one 
physical might be used to increase total protection. A line should be drawn 
around thepremises on the site plan, that needs to be protected. A perimeter in 
this context is something that povides a physical barrier between the 
organization and the outside world:  walls, doors, windows, gates, floors, fixed
ceilings (false ceilings hide a multitude of threats), skylights, etc. Special
attention should also be gi ven to lifts and lift shafts, risers, maintenance 
and access shafts, etc. A comprehensive risk assessment should be carried out to
identify the weaknesses, vulnerabilities or gaps in this perimeter.
The following controls should form part of the implemented security perimeter:
\begin{itemize} 
    \item The perimeter is defined in a document, and staff are aware of what 
    and where it is.

    \item The perimeter should be physically sound. There should be no gaps.
    External walls should be of solid construction, and doors should be protected
    for unauthorized access.

    \item There should be a staffed reception to prevent unauthorized access.

    \item Physical barriers should be extended from real floor to real ceiling
    (ie below and above any false floor or false ceiling, particularly those 
    installed to provide effective ducting for cabling) to prevent unauthorized 
    entry or environmental contamination such as that caused by fire or flood.

    \item Fire doors should open out, should slam shut, and should be alarmed.
    This should be advertised to prevent false alarms, there should be CCTV to
    watch for this.

    \item Appropriate intruder detection systems should be professionally
    installed and maintained. All external doors and accessible windows
    should be covered and unoccupied areas should always be alarmed.
    Protocol for what to do when alarm goes of should be a part of ISMS.
\end{itemize}

\subsection{Physical Entry Controls}
Secure areas should be protected with appropriate entry controls. ISO27002
recommends specific controls:
\begin{itemize}
    \item Visitors should be supervised or cleared in advance. Records with
    time stamps of visitors should be kept. 

    \item Outsourced security services (Securitas) should be vetted 
    independently, and should recive trainiing on the organizations security 
    procedures.

    \item If access is granted remotely there should be a communication device.

    \item There should be a auditable trail for usage of key cards and pins.

    \item Personel should be requierd to wear some visible identification.

    \item All staff should that might encounter Visitors should be trained so that
    it is difficult for a social engineer to bypass physical security controls.

    \item Access rights to secure areas should regularly be reviewed, updated and,
    where necessary, revoked, and a record of the forum’s review should form 
    part of the ISMS documentation.

\end{itemize}

\subsection{Securing Offices, Rooms and Facilities}
A secure room may cont ain lockable cabinets or safes. Secure rooms could be any
rooms within the pr emises but  will certainly include server rooms,
telecommunications rooms  and plant  (power and air-conditioning) rooms. Some 
other (such as accounts or HR, or directors’ offices) might also need to be 
secured. CEOs’ offices should also be treated as secure rooms. Secure  area 
design should take account of the possibility of damage from fire, flood, 
explosion, civil unrest and other forms of natural or humancreated disaster.
The controls that ISO27002 recommends should be considered and, if  appropriate,
implemented include the following:
\begin{itemize}
    \item  Key storage areas and keyed entrance areas should be sited to
     avoid access by unauthorized persons and by the public.

     \item Bulding should be inobtrusive, and give little indication of their 
     precense or purpose.

     \item Doors and windows should be locked when the room is unattended.
     Burglar bars should be considered for ground floor.

     \item Secure areas should be seperated from third-parties. 

     \item Internal directories or phone books or other information about 
     locations of secure areas should not be accessible by the public.

     \item Hazardous material should not be bulk stored in a secure area.

     \item Backup equipment should not be stored with the equipment that they 
     will back up.
\end{itemize}


